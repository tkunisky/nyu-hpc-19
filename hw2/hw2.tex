\documentclass[11pt]{article}

% Packages
\usepackage{courier}
\usepackage{subfig}
\usepackage{mathrsfs}
\usepackage{makecell}
\usepackage{amscd}
\usepackage{amsmath}
\usepackage{amssymb}
\usepackage{amstext}
\usepackage[thmmarks,amsthm,amsmath]{ntheorem}
\usepackage{apptools}
\usepackage{bbold}
\usepackage{bm}
\usepackage{booktabs}
\usepackage{color}
\usepackage{easybmat}
\usepackage{etex}
\usepackage{framed}
\usepackage[dvips,letterpaper,margin=1in]{geometry}
\usepackage{graphicx}
\usepackage{hyperref}
\usepackage{listings}
\usepackage[noabbrev,capitalize]{cleveref}
% \usepackage[lucidasmallscale=true]{lucimatx}
\usepackage{mathtools}
\usepackage{colonequals}
\usepackage{longtable}
\usepackage{rotating}
\usepackage{setspace}
\usepackage{tabu}
\usepackage{verbatim}


\newcommand{\footremember}[2]{%
    \footnote{#2}
    \newcounter{#1}
    \setcounter{#1}{\value{footnote}}%
}
\newcommand{\footrecall}[1]{%
    \footnotemark[\value{#1}]%
  }


% 
% Theorems
%

\lstset{basicstyle=\footnotesize\ttfamily\singlespacing,breaklines=true}

  
\newtheorem{claim}{Claim}[section]

%\definecolor{gray}{rgb}{0.5, 0.5, 0.5}
%\renewcommand*\FrameCommand{{\color{gray}\vrule width 5pt \hspace{10pt}}}
 
\newtheorem{theorem}{Theorem}[section]
\newtheorem*{remark}{Remark}
\newtheorem{lemma}[theorem]{Lemma}
\newtheorem{question}[theorem]{Question}
\newtheorem{problem}[theorem]{Problem}
\newtheorem{example}[theorem]{Example}
\newtheorem{proposition}[theorem]{Proposition}
\newtheorem{definition}[theorem]{Definition}
\newtheorem{corollary}[theorem]{Corollary}
\AtAppendix{\renewtheorem{corollary}[theorem]{Corollary}}
\newtheorem{conjecture}[theorem]{Conjecture}

\crefname{theorem}{Theorem}{Theorems}
\crefname{proposition}{Proposition}{Propositions}
\crefname{lemma}{Lemma}{Lemmas}

\theoremstyle{plain} % just in case the style had changed
\newcommand{\thistheoremname}{}
\newtheorem*{genericthm}{\thistheoremname}
\newenvironment{namedtheorem}[1]
  {\renewcommand{\thistheoremname}{#1}%
   \begin{genericthm}}
  {\end{genericthm}}

%
% Commands
%

\newcommand{\what}{\widehat}

% Figures
\newcommand{\fig}[1]{(figure \ref{#1})}
\newcommand{\FIG}[1]{figure \ref{#1}}

\DeclareSymbolFont{bbold}{U}{bbold}{m}{n}
\DeclareSymbolFontAlphabet{\mathbbold}{bbold}

% Names
\newcommand{\Lovasz}{Lov\'{a}sz}
\newcommand{\Mobius}{M\"{o}bius}
\newcommand{\Holder}{H\"{o}lder}
\newcommand{\Rouche}{Rouch\'{e}}
\newcommand{\Ito}{It\={o}}
\newcommand{\Kondo}{Kond\^{o}}
\newcommand{\Levy}{L\'{e}vy}
\newcommand{\Cramer}{Cram\'{e}r}
\newcommand{\Godel}{G\"{o}del}
\newcommand{\Carath}{Carath\'{e}odory}
\newcommand{\Caratheodory}{Carath\'{e}odory}
\newcommand{\Schlafli}{Schl\"{a}fli}
\newcommand{\Hopital}{H\^{o}pital}

% Random
\newcommand{\lvec}{\overrightarrow}
\newcommand{\ra}{\rangle}
\newcommand{\la}{\langle}

% Disjoint union
\makeatletter
\def\moverlay{\mathpalette\mov@rlay}
\def\mov@rlay#1#2{\leavevmode\vtop{%
   \baselineskip\z@skip \lineskiplimit-\maxdimen
   \ialign{\hfil$\m@th#1##$\hfil\cr#2\crcr}}}
\newcommand{\charfusion}[3][\mathord]{
    #1{\ifx#1\mathop\vphantom{#2}\fi
        \mathpalette\mov@rlay{#2\cr#3}
      }
    \ifx#1\mathop\expandafter\displaylimits\fi}
\makeatother

\newcommand{\cupdot}{\charfusion[\mathbin]{\cup}{\cdot}}
\newcommand{\bigcupdot}{\charfusion[\mathop]{\bigcup}{\cdot}}

\renewcommand{\dim}{\mathsf{dim}}

% Blackboard bold
\newcommand{\RR}{\mathbb{R}}
\newcommand{\QQ}{\mathbb{Q}}
\newcommand{\NN}{\mathbb{N}}
\newcommand{\ZZ}{\mathbb{Z}}
\newcommand{\DD}{\mathbb{D}}
\newcommand{\HH}{\mathbb{H}}
\newcommand{\CC}{\mathbb{C}}
\newcommand{\PP}{\mathbb{P}}
\newcommand{\EE}{\mathbb{E}}
\renewcommand{\AA}{\mathbb{A}}
\newcommand{\FF}{\mathbb{F}}
\newcommand{\MM}{\mathbb{M}}
\renewcommand{\SS}{\mathbb{S}}
\newcommand{\Fp}{\FF_p}
\newcommand{\TrivGp}{\mathbbold{1}}
\newcommand{\One}{\mathbbold{1}}

\newcommand{\RP}{\RR\mathrm{P}}
\newcommand{\CP}{\CC\mathrm{P}}

% Vector bold
\newcommand{\bq}{\bm{q}}
\newcommand{\nn}{\bm{n}}
\newcommand{\vv}{\bm{v}}
\newcommand{\ww}{\bm{w}}
\newcommand{\xx}{\bm{x}}
\newcommand{\yy}{\bm{y}}
\newcommand{\one}{\bm{1}}

% Other fonts
\newcommand{\fp}{\mathfrak{p}}
\newcommand{\fq}{\mathfrak{q}}
\newcommand{\fg}{\mathfrak{g}}
\newcommand{\fh}{\mathfrak{h}}
\newcommand{\fa}{\mathfrak{a}}
\newcommand{\fb}{\mathfrak{b}}
\newcommand{\fc}{\mathfrak{c}}
\newcommand{\fm}{\mathfrak{m}}
\renewcommand{\sl}{\mathfrak{sl}}
\newcommand{\so}{\mathfrak{so}}
\newcommand{\gl}{\mathfrak{gl}}
\renewcommand{\sp}{\mathfrak{sp}}
\newcommand{\sA}{\mathcal{A}}
\newcommand{\sB}{\mathcal{B}}
\newcommand{\sC}{\mathcal{C}}
\newcommand{\sD}{\mathcal{D}}
\newcommand{\sE}{\mathcal{E}}
\newcommand{\sF}{\mathcal{F}}
\newcommand{\sG}{\mathcal{G}}
\newcommand{\sH}{\mathcal{H}}
\newcommand{\sI}{\mathcal{I}}
\newcommand{\sL}{\mathcal{L}}
\newcommand{\sM}{\mathcal{M}}
\newcommand{\sN}{\mathcal{N}}
\newcommand{\sO}{\mathcal{O}}
\newcommand{\sP}{\mathcal{P}}
\newcommand{\sR}{\mathcal{R}}
\newcommand{\sS}{\mathcal{S}}
\newcommand{\sT}{\mathcal{T}}
\newcommand{\sU}{\mathcal{U}}
\newcommand{\sV}{\mathcal{V}}
\newcommand{\sX}{\mathcal{X}}
\newcommand{\sY}{\mathcal{Y}}

% Plain text
\renewcommand{\Re}{\mathrm{Re}}
\renewcommand{\Im}{\mathrm{Im}}

\newcommand{\Tr}{\mathrm{Tr}}

\newcommand{\img}{\mathrm{img}}
\renewcommand{\ker}{\mathsf{ker}}
\newcommand{\sgn}{\mathsf{sgn}}

\newcommand{\ch}{\mathrm{char}}
\newcommand{\Res}{\mathrm{Res}}
\newcommand{\ord}{\mathrm{ord}}
\newcommand{\cont}{\mathrm{cont}}
\newcommand{\ab}{\mathrm{ab}}
\newcommand{\Orb}{\mathrm{Orb}}
\newcommand{\Syl}{\mathrm{Syl}}
\newcommand{\Irr}{\mathrm{Irr}}
\newcommand{\Frac}{\mathrm{Frac}}
\newcommand{\sep}{\mathrm{sep}}
\newcommand{\per}{\mathrm{per}}

% Groups
\newcommand{\GL}{\mathrm{GL}}
\newcommand{\PGL}{\mathrm{PGL}}
\newcommand{\PSL}{\mathrm{PSL}}
\newcommand{\SL}{\mathrm{SL}}
\newcommand{\oO}{\mathrm{O}}
\newcommand{\SO}{\mathrm{SO}}
\newcommand{\PSO}{\mathrm{PSO}}
\newcommand{\Sp}{\mathrm{Sp}}
\newcommand{\PSp}{\mathrm{PSp}}
\newcommand{\U}{\mathrm{U}}
\newcommand{\SU}{\mathrm{SU}}
\newcommand{\PSU}{\mathrm{PSU}}

% Parentheses
\newcommand{\lgndr}[2]{\ensuremath{\left(\frac{#1}{#2}\right)}}

% Mappings
\newcommand{\iso}{\cong}
\newcommand{\eqdf}{\stackrel{\mathrm{df}}{=}}
\newcommand{\eqd}{\stackrel{\mathrm{(d)}}{=}}
\newcommand{\eqqu}{\stackrel{\mathrm{?}}{=}}
\newcommand{\xto}{\xrightarrow}
\newcommand{\dto}{\Rightarrow}
\newcommand{\into}{\hookrightarrow}
\newcommand{\xinto}{\xhookrightarrow}
\newcommand{\onto}{\twoheadrightarrow}
\newcommand{\xonto}{xtwoheadrightarrow}
\newcommand{\isoto}{\xto{\sim}}
\newcommand{\upto}{\nearrow}
\newcommand{\downto}{\searrow}

% Convenience
\newcommand{\Implies}{\ensuremath{\Rightarrow}}
\newcommand{\ImpliedBy}{\ensuremath{\Leftarrow}}
\newcommand{\Iff}{\ensuremath{\Leftrightarrow}}

\newcommand{\Pfright}{\ensuremath{(\Rightarrow):\hs}}
\newcommand{\Pfleft}{\ensuremath{(\Leftarrow):\hs}}

\newcommand{\sm}{\ensuremath{\setminus}}

\newcommand{\tab}[1]{(table \ref{#1})}
\newcommand{\TAB}[1]{table \ref{#1}}

\newcommand{\precode}[1]{\textbf{\footnotesize #1}}
\newcommand{\code}[1]{\texttt{\footnotesize #1}}

\newcommand{\sectionline}{
  \nointerlineskip \vspace{\baselineskip}
  \hspace{\fill}\rule{0.35\linewidth}{.7pt}\hspace{\fill}
  \par\nointerlineskip \vspace{\baselineskip}
}


%
% Misc
%

% \parskip0em
% \linespread{1.05}
% \widowpenalty10000
% \clubpenalty10000


\renewcommand{\vec}{\mathsf{vec}}
\renewcommand{\sep}{\mathsf{sep}}
\newcommand{\rowvec}{\mathsf{rowvec}}
\newcommand{\row}{\mathsf{row}}
\newcommand{\HC}{\mathsf{HC}}
\newcommand{\CHC}{\mathsf{CHC}}
\newcommand{\SK}{\mathsf{SK}}
\newcommand{\SDP}{\mathsf{SDP}}
\newcommand{\SOS}{\mathsf{SOS}}
\newcommand{\conv}{\mathsf{conv}}
\newcommand{\Gram}{\mathsf{Gram}}
\newcommand{\F}{\mathsf{F}}
\newcommand{\M}{\mathsf{M}}
\newcommand{\FP}{\mathsf{FP}}
\newcommand{\PE}{\mathsf{PE}}
\newcommand{\PM}{\mathsf{PM}}
\newcommand{\CUT}{\mathsf{CUT}}
\newcommand{\PS}{\mathsf{PS}}
\newcommand{\PSF}{\mathsf{PSF}}
\newcommand{\srg}{\mathsf{srg}}
\newcommand{\sym}{\mathsf{sym}}
\newcommand{\herm}{\mathsf{herm}}
\newcommand{\mat}{\mathsf{mat}}
\newcommand{\antisym}{\mathsf{antisym}}
\newcommand{\tEE}{\tilde{\mathbb{E}}}
\newcommand{\sfP}{\mathsf{P}}
\DeclareMathOperator*{\argmin}{\arg\!\min}
\newcommand{\diag}{\mathsf{diag}}
\newcommand{\op}{\mathsf{op}}
\newcommand{\rank}{\mathsf{rank}}
\newcommand{\rk}{\mathsf{rk}}
\newcommand{\rkeff}{\mathrm{rk}_{\mathrm{eff}}}
\newcommand{\GOE}{\mathsf{GOE}}
\newcommand{\ssG}{\mathsf{G}}
\newcommand{\bA}{\bm A}
\newcommand{\bB}{\bm B}
\newcommand{\bC}{\bm C}
\newcommand{\bD}{\bm D}
\newcommand{\bE}{\bm E}
\newcommand{\bF}{\bm F}
\newcommand{\bG}{\bm G}
\newcommand{\bH}{\bm H}
\newcommand{\bL}{\bm L}
\newcommand{\bM}{\bm M}
\newcommand{\bN}{\bm N}
\newcommand{\bP}{\bm P}
\newcommand{\bQ}{\bm Q}
\newcommand{\bR}{\bm R}
\newcommand{\bS}{\bm S}
\newcommand{\bT}{\bm T}
\newcommand{\bU}{\bm U}
\newcommand{\bV}{\bm V}
\newcommand{\bW}{\bm W}
\newcommand{\bX}{\bm X}
\newcommand{\bY}{\bm Y}
\newcommand{\bZ}{\bm Z}
\newcommand{\ba}{\bm a}
\newcommand{\bb}{\bm b}
\newcommand{\bc}{\bm c}
\newcommand{\bd}{\bm d}
\newcommand{\be}{\bm e}
\newcommand{\bg}{\bm g}
\newcommand{\bh}{\bm h}
\newcommand{\bi}{\bm i}
\newcommand{\bj}{\bm j}
\newcommand{\br}{\bm r}
\newcommand{\bs}{\bm s}
\newcommand{\hbr}{\hat{\bm r}}
\newcommand{\hbs}{\hat{\bm s}}
\newcommand{\bt}{\bm t}
\newcommand{\bu}{\bm u}
\newcommand{\bv}{\bm v}
\newcommand{\bw}{\bm w}
\newcommand{\bx}{\bm x}
\newcommand{\by}{\bm y}
\newcommand{\bz}{\bm z}

\newcommand{\T}{\mathsf{T}}
\newcommand{\Paley}{\mathsf{Paley}}
\newcommand{\Johnson}{\mathsf{Johnson}}
\newcommand{\OA}{\mathsf{OA}}
\newcommand{\VO}{\mathsf{VO}}
\newcommand{\RSHCD}{\mathsf{RSHCD}}

\newcommand{\fB}{\mathscr{B}}
\newcommand{\fC}{\mathscr{C}}
\newcommand{\fE}{\mathscr{E}}

\newcommand{\odd}{\mathsf{odd}}

\newcommand{\GramSDP}{\mathsf{GramSDP}}
\newcommand{\ptop}{\mathsf{\Gamma}}

\newcommand{\pert}{\mathsf{pert}}

\DeclareRobustCommand{\bmrob}[1]{\bm{#1}}
\pdfstringdefDisableCommands{%
  \renewcommand{\bmrob}[1]{#1}%
}

\title{High Performance Computing: Homework 2} 
\author{Dmitriy (Tim) Kunisky [dk3105]}
\date{}

\begin{document}

\maketitle

\begin{itemize}
\item Processor: Intel(R) Core(TM) i5-5257U CPU @ 2.70GHz
    \begin{itemize}
    \item Maximum flop rate: 19.76 Gflop/s, or 4.94 Gflop/s per core
    \item Maximum main memory bandwidth: 25.6 GB/s
    \end{itemize}
\item Compiler: Apple LLVM version 9.1.0 (clang-902.0.39.2)
\end{itemize}

\subsection*{Problem 2: Matrix-Matrix Multiplication}

\paragraph{Loop orderings.}
The fastest loop order appears to be $(j, p, i)$, corresponding
to traversing columns with temporal locality as much as possible (since traversing
the $i$ index traverses a column of both $\bm A$ and $\bm C$, traversing the
$p$ index traverses a column of $\bm B$ and a row of $\bm A$, and traversing the
$j$ index traverses a row of both $\bm B$ and $\bm C$).
We then expect the most harmful changes to this ordering to be those that move
column traversals earlier in the loop order.
Thus, the costliest ordering should be the reverse, $(i, p, j)$, while orderings
like $(j, i, p)$ should be intermediate.
The table below illustrates that this is indeed the case.

\begin{center}
\small
\begin{tabular}{r|rr|rr|rr}
  & \multicolumn{2}{c|}{Ordering $(j, p, i)$} & \multicolumn{2}{c|}{Ordering $(j, i, p)$} & \multicolumn{2}{c}{Ordering $(i, p, j)$} \\
  \hline
  Dimension & \makecell{Flop Rate \\ (Gflop/s)} & \makecell{Bandwidth \\ (GB/s)} & \makecell{Flop Rate \\ (Gflop/s)} & \makecell{Bandwidth \\ (GB/s)} & \makecell{Flop Rate \\ (Gflop/s)} & \makecell{Bandwidth \\ (GB/s)} \\
  \hline
  16 & 5.93 & 94.82 & 2.53 & 40.51 & 2.62 & 41.94 \\
  112 & 9.62 & 154.00 & 1.83 & 29.23 & 1.54 & 24.70 \\
  304 & 7.79 & 124.64 & 0.91 & 26.98 & 0.77 & 12.34 \\
  640 & 5.66 & 90.64 & 0.82 & 13.15 & 0.65 & 10.43 \\
  976 & 4.38 & 70.05 & 0.87 & 13.90 & 0.38 & 6.05 \\
  1408 & 4.17 & 66.71 & 0.69 & 11.12 & 0.19 & 3.07 \\
  \hline
\end{tabular}
\end{center}

\paragraph{Block size tuning.}

It appears that the optimal block size is around 60.
One possible justification for this is that the processor used for these tests has a Level 1 cache of size 64KB.
If the block size is $B$, then the number of bytes required to store two $B \times B$ arrays of doubles (the size of the data on which arithmetic is being performed in one block matrix operation) is $2 \cdot 8 \cdot B^2$, which is smaller than 64KB only while $B \leq 60$ (over values of $B$ divisible by 4).
The figures in the table below support this heuristic calculation.

\begin{center}
\small
\begin{tabular}{r|rr|rr|rr}
  & \multicolumn{2}{c|}{$B = 40$} & \multicolumn{2}{c|}{$B = 60$} & \multicolumn{2}{c}{$B = 80$} \\
  \hline
  Dimension & \makecell{Flop Rate \\ (Gflop/s)} & \makecell{Bandwidth \\ (GB/s)} & \makecell{Flop Rate \\ (Gflop/s)} & \makecell{Bandwidth \\ (GB/s)} & \makecell{Flop Rate \\ (Gflop/s)} & \makecell{Bandwidth \\ (GB/s)} \\
  \hline
  240 & 4.69 & 75.05 & 8.95 & 143.26 & 9.12 & 145.91 \\
  480 & 5.37 & 85.93 & 8.79 & 140.78 & 7.92 & 126.78 \\
  720 & 5.28 & 84.55 & 8.68 & 138.85 & 8.38 & 134.07 \\
  960 & 5.26 & 84.20 & 8.80 & 140.82 & 6.52 & 104.37 \\
  1200 & 5.41 & 86.48 & 8.33 & 133.21 & 6.84 & 109.41 \\
  1440 & 5.19 & 82.97 & 8.56 & 137.01 & 7.44 & 119.04 \\
  1680 & 5.31 & 85.03 & 8.16 & 130.55 & 6.87 & 109.91 \\
  \hline
\end{tabular}
\end{center}

\paragraph{Parallelization.}

Finally, we parallelize our implementation by simply using the parallelized for loop directive of OpenMP on the top-level block iteration.
This roughly doubles the flop rate, which corresponds to this processor having two cores.
With this optimization, the matrix multiplication reaches a flop rate of slightly more than 17 Gflop/s, which is approximately 86\% of the peak flop rate of 19.76 Gflop/s.

\begin{center}
\small
\begin{tabular}{r|rr|rr}
  & \multicolumn{2}{c|}{$B = 60$, Serial} & \multicolumn{2}{c}{$B = 60$, Parallel} \\
  \hline
  Dimension & \makecell{Flop Rate \\ (Gflop/s)} & \makecell{Bandwidth \\ (GB/s)} & \makecell{Flop Rate \\ (Gflop/s)} & \makecell{Bandwidth \\ (GB/s)} \\
  \hline
  240 & 8.95 & 143.26 & 19.68 & 314.86 \\
  480 & 8.79 & 140.78 & 18.41 & 310.60 \\
  720 & 8.68 & 138.85 & 18.68 & 298.84 \\
  960 & 8.80 & 140.82 & 17.01 & 272.16 \\
  1200 & 8.33 & 133.21 & 17.85 & 285.60 \\
  1440 & 8.56 & 137.01 & 17.59 & 281.39 \\
  1680 & 8.16 & 130.55 & 17.24 & 275.79 \\
  \hline
\end{tabular}
\end{center}

\subsection*{Problem 4: Jacobi/Gauss-Seidel Smoothing}

We present timing results for varying problem sizes of the Jacobi and Gauss-Seidel smoothers, for one, two, and four threads.
The machine used has two cores, so we expect to see a benefit from two threads versus one, but no additional benefit from four threads versus two.
The results below support this claim; for large problems, the speed roughly doubles from moving from one to two threads, but there is no effect from moving from two to four threads.

\begin{center}
\small
\begin{tabular}{r|rrr|rrr}
  & \multicolumn{3}{c|}{Jacobi} & \multicolumn{3}{c}{Gauss-Seidel} \\
  \hline
  N & 1 Thread & 2 Threads & 4 Threads & 1 Thread & 2 Threads & 4 Threads \\
  \hline
  50 & 0.15 & 0.19 & 0.18 & 0.25 & 0.20 & 0.28 \\
  100 & 0.55 & 0.34 & 0.35 & 0.90 & 0.57 & 0.65 \\
  150 & 1.35 & 0.68 & 0.84 & 2.03 & 1.10 & 1.22 \\
  200 & 2.28 & 1.13 & 1.51 & 3.54 & 2.01 & 2.20 \\
  300 & 5.03 & 2.70 & 3.39 & 7.89 & 4.18 & 4.38 \\
  400 & 10.46 & 7.86 & 7.64 & 17.31 & 7.74 & 8.05 \\
  \hline
\end{tabular}
\end{center}

\noindent
(Note that running the parallel version of either algorithm is implemented by passing an arbitrary third argument to the executable, as shown in the top comment of either source file.)

\end{document}

%%% Local Variables:
%%% mode: latex
%%% TeX-master: "SemidefiniteProgramming_Kunisky"
%%% End:
